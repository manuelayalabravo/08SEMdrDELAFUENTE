% Options for packages loaded elsewhere
\PassOptionsToPackage{unicode}{hyperref}
\PassOptionsToPackage{hyphens}{url}
%
\documentclass[
  american,
]{article}
\usepackage{lmodern}
\usepackage{amssymb,amsmath}
\usepackage{ifxetex,ifluatex}
\ifnum 0\ifxetex 1\fi\ifluatex 1\fi=0 % if pdftex
  \usepackage[T1]{fontenc}
  \usepackage[utf8]{inputenc}
  \usepackage{textcomp} % provide euro and other symbols
\else % if luatex or xetex
  \usepackage{unicode-math}
  \defaultfontfeatures{Scale=MatchLowercase}
  \defaultfontfeatures[\rmfamily]{Ligatures=TeX,Scale=1}
\fi
% Use upquote if available, for straight quotes in verbatim environments
\IfFileExists{upquote.sty}{\usepackage{upquote}}{}
\IfFileExists{microtype.sty}{% use microtype if available
  \usepackage[]{microtype}
  \UseMicrotypeSet[protrusion]{basicmath} % disable protrusion for tt fonts
}{}
\makeatletter
\@ifundefined{KOMAClassName}{% if non-KOMA class
  \IfFileExists{parskip.sty}{%
    \usepackage{parskip}
  }{% else
    \setlength{\parindent}{0pt}
    \setlength{\parskip}{6pt plus 2pt minus 1pt}}
}{% if KOMA class
  \KOMAoptions{parskip=half}}
\makeatother
\usepackage{xcolor}
\IfFileExists{xurl.sty}{\usepackage{xurl}}{} % add URL line breaks if available
\IfFileExists{bookmark.sty}{\usepackage{bookmark}}{\usepackage{hyperref}}
\hypersetup{
  pdftitle={Informe de seminario 8},
  pdfauthor={Manuel Ayala},
  pdflang={en-US},
  hidelinks,
  pdfcreator={LaTeX via pandoc}}
\urlstyle{same} % disable monospaced font for URLs
\usepackage[margin=1in]{geometry}
\usepackage{graphicx}
\makeatletter
\def\maxwidth{\ifdim\Gin@nat@width>\linewidth\linewidth\else\Gin@nat@width\fi}
\def\maxheight{\ifdim\Gin@nat@height>\textheight\textheight\else\Gin@nat@height\fi}
\makeatother
% Scale images if necessary, so that they will not overflow the page
% margins by default, and it is still possible to overwrite the defaults
% using explicit options in \includegraphics[width, height, ...]{}
\setkeys{Gin}{width=\maxwidth,height=\maxheight,keepaspectratio}
% Set default figure placement to htbp
\makeatletter
\def\fps@figure{htbp}
\makeatother
\setlength{\emergencystretch}{3em} % prevent overfull lines
\providecommand{\tightlist}{%
  \setlength{\itemsep}{0pt}\setlength{\parskip}{0pt}}
\setcounter{secnumdepth}{-\maxdimen} % remove section numbering
\usepackage{pgf,tikz}
\ifxetex
  % Load polyglossia as late as possible: uses bidi with RTL langages (e.g. Hebrew, Arabic)
  \usepackage{polyglossia}
  \setmainlanguage[variant=american]{english}
\else
  \usepackage[shorthands=off,main=american]{babel}
\fi
\usepackage[]{biblatex}
\addbibresource{./references.bib}

\title{Informe de seminario 8}
\usepackage{etoolbox}
\makeatletter
\providecommand{\subtitle}[1]{% add subtitle to \maketitle
  \apptocmd{\@title}{\par {\large #1 \par}}{}{}
}
\makeatother
\subtitle{\textbf{Econometric Modeling and Solving Social Problems}}
\author{Manuel Ayala}
\date{14 julio 2020}

\begin{document}
\maketitle

\hypertarget{descripciuxf3n-del-trabajo-presentado}{%
\subsection{Descripción del trabajo
presentado}\label{descripciuxf3n-del-trabajo-presentado}}

El seminario realizado por el Dr.~De La Fuente presentó modelos
econométricos con carácter social. El desarrollo de la exposición
consideró tres casos de estudios definidos como:

\begin{itemize}
\tightlist
\item
  Caso 1 Analysis of the factor of Chilean city hall using econometric
  modeling and stochastic frontier.
\end{itemize}

Considerando el índice de calidad de vida \((QoL)\) sectorizado por
comunas de Chile, publicado por la PUC en el año 2018, el cual
consideraba variables cualitativas y cuantitativas de la calidad como
conectividad, habitabilidad, salud, condiciones sociocultirales y otras,
se desarrollo estudio que buscaba determinar el grado de influencia real
de la calidad de vida por comuna y estimar la eficiencia de las
municipalidades a partir de los indicadores de calidad de vida como una
calificación de vida urbana. Esto se relizó basado en estadísticas
descriptivas y el modelo de regresión \emph{stepwise}, análisis de
cluster y modelos estocásticos de frontera \emph{Cobb Douglas y
Translogarithmic}. el resultado del modelo y el índice de eficiencia
llevo a determinar que la comuna con mayor índice de eficiencia en Chile
es Punta Arenas \autocite{hanns_de_la_fuente-mella_econometric_2020}.

\begin{itemize}
\tightlist
\item
  Caso 2 Forecasting performance measure for traffic safety using
  deterministic and stochastic models.
\end{itemize}

A partir de la ley Moving Ahead Progress in 21st Century (MAP21) en el
estado de Nevada, USA . Se realizó estudio que consideró el conteo del
tráfico vehicular, se generaron bases de datos y posterior modelado
determinístico y estocástico \autocite{paz_forecasting_2015} para
pronosticar el desempeño de seguridad de las Autopistas del Estado y
llevar a 0 la posibilidad de muertes por accidentes de tránsito, reducir
el número de accidentes graves. A partir de las bases de datos se
realizarón varios modelos estadísticos determinísticos de predicción de
accidentes, donde el modelo que mejor se ajustó fue el modelo
\emph{Winter Additive}. luego, se realizó predicción de las variables a
través de modelos estocásticos estacionarios ARIMA, llegando a la
conclusión que con los datos existentes el mejor ajuste de las variales
\emph{número de muertes y accidentes graves} se lograva con RMSE (error
medio cuadrático) de 7.756 y 30.111 respectivamente
\autocite{hanns_de_la_fuente-mella_econometric_2020}.

\begin{itemize}
\tightlist
\item
  Caso 3 Gobiernos corporativos y asimetrías de información.
  Modelamiento econométrico del Spread (Bid-Ask) para una muestra de
  Empresas Chilenas.
\end{itemize}

En los mercados de capitales el manejo de la información y la
transparencia de esta genera asimetrías en las transacciones en
correlación con el \emph{spread} generado en la tansacción. esta
información manejada por los corredores de bolsa, hace que en ocaciones
se desestabilice por completo el mercado accionario y se produzcan
efectos que afectan a todo el sistema económico y social de una nación.
se presentó modelo econométrico multivariante qeu determina el efecto de
asimetrias de información asociado al tamaño del corredor y el spread
generado en la transacción como también el efecto que se puede
interpretar como una acción monopólica en el manejo del spread. Este
modelo se genero a partir de información compartida por la bolsa de
valores de Santiago.

\hypertarget{anuxe1lisis-de-la-literatura-relacionada}{%
\subsection{Análisis de la literatura
relacionada}\label{anuxe1lisis-de-la-literatura-relacionada}}

Luego de realizar búsqueda de literatura en la base artículos
científicos Web of Science y utilizando la herramienta bibliometrix se
obtuvieron estadísticos de la producción científica respecto al estudio
presentado, la base de busqueda se basó utilizando las keywords:
``econometric+model'', filtrado por la categoria social science
mathematical method. La literatura relacionada encontrada de acuerdo a
frase clave, mostró producción científica desde el año 1975, con
producción de 314 artículos, y una cantidad aproximada de 800 autores.

\includegraphics{Figs/unnamed-chunk-2-1.pdf}
\includegraphics{Figs/unnamed-chunk-2-2.pdf}
\includegraphics{Figs/unnamed-chunk-2-3.pdf}
\includegraphics{Figs/unnamed-chunk-2-4.pdf}
\includegraphics{Figs/unnamed-chunk-2-5.pdf}

\includegraphics{Figs/unnamed-chunk-3-1.pdf}

de los gráficos anteriores se destaca una mayor producción científica en
Estados Unidos comparatiovamente con el resto del mundo, además, se
verifica una producción científica en aumento de acuerdo a la cantidad
de trabajos publicados.

\hypertarget{descripciuxf3n-del-marco-conceptual}{%
\subsection{Descripción del marco
conceptual}\label{descripciuxf3n-del-marco-conceptual}}

el trabajo del profesor De La Fuente y su caracter social de la
investigación expuesta, esta basado en la rama de la economía donde su
impacto de modelación afecta socialmente de forma directa como lo
expuesto en el caso1, la eficiencia de las municipalidades o derivada
del juego bursatil de la bolsa de valores del caso 3. Este modelado
matemático de los sistemas económicos de un grupo de personas o paises,
en diferentes actividades económicas con datos observados se identifica
como econometría y se define como: El análisis cuantitativo de fenómenos
económicos actuales, basado en el desarrollo congruente de teoría y
observaciones, y relacionado por métodos apropiados de
inferencia\autocite{samuelson_report_1954-1}.

los modelos econométricos presentados se enmarcan en el análsis
predictivo con técnicas estadísticas de modelación, donde basados en
datos históricos no modelados de variables se infirieron acontecimientos
futuros basados en el ajuste de algún modelo matemático.

En el caso 1 presentado, se utilizaron en la metodología de modelado
varias técnicas matemáticas como estadística descrptiva, ajuste por
regresión stepwise, análisis de cluster y Stochastic Frontier Analysis
con dos modelos:

\begin{itemize}
\item
  \textbf{Cobb-Douglas} \[\ln q_{it}=X_{it}\beta +v_{it}-u_{it}\]
\item
  \textbf{Translogarithmic}
  \[\ln q_{it}= \beta +\sum_{i=1}^n \beta \ln X_i+0.5\sum_{i=1}^n\sum_{j=1}^n \beta \ln X_i \ln X_j\]
\end{itemize}

En el caso 2 se utilizó como modelado de la serie al modelo ARIMA que es
un modelo autorregresivo integrado de promedio móvil (acrónimo del
inglés autoregressive integrated moving average).

\autocite{casimiro_alisis_nodate}La construcción general de modelos
ARIMA se lleva a cabo de forma iterativa mediante un proceso en el que
se pueden distinguir cuatro etapas: a) Identificación. Utilizando los
datos y/o cualquier tipo de información disponible sobre cómo ha sido
generada la serie, se intenta sugerir una subclase de modelos
ARIMA\((p, d, q)\) que merezca la pena ser investigada. El objetivo es
determinar los órdenes \(p,d,q\) que parecen apropiados para reproducir
las características de la serie bajo estudio y si se incluye o no la
constante \(δ\). En esta etapa es posible identificar m ́as de un modelo
candidato a haber podido generar la serie. b) Estimación. Usando de
forma eficiente los datos se realiza inferencia sobre los parámetros
condicionada a que el modelo investigado sea apropiado. Dado un
determinado proceso propuesto, se trata de cuantificar los parámetros
del mismo, \(θ1,...θq, φ1,...φp , σ2\) y, en su caso, \(δ\). c)
Validación. Se realizan contrastes de diagnóstico para comprobar si el
modelo se ajusta a los datos, o, si no es así, revelar las posibles
discrepancias del modelo propuesto para poder mejorarlo. d) Predicción.
Obtener pronósticos en términos probabilísticos de los valores futuros
de la variable. En esta etapa se tratará también de evaluar la capacidad
predictiva del modelo.

El caso 3, basó su marco teórico en la asimetría de la información, la
cual en el mercado accionario de divisas es medida a través del
\emph{spread} del valor entre la compra y la venta de los precios
accionarios. En un mercado plenamente activo se formaría naturalmente un
precio de equilibrio entre la oferta y la demanda, cuando ello no ocurre
surge el precio \emph{Bid} y el precio
\emph{Ask}\autocite{hanns_de_la_fuente-mella_econometric_2020}, de
acuerdo al siguiente diagrama:

\tikzset{every picture/.style={line width=0.75pt}}

\begin{tikzpicture}[x=0.75pt,y=0.75pt,yscale=-1,xscale=1]
%uncomment if require: \path (0,300); %set diagram left start at 0, and has height of 300

%Shape: Axis 2D [id:dp19527234072026634] 
\draw  (151.2,238.94) -- (430.21,238.65)(178.8,44.76) -- (179.14,260.49) (423.2,233.66) -- (430.21,238.65) -- (423.22,243.66) (173.81,51.77) -- (178.8,44.76) -- (183.81,51.76)  ;
%Shape: Free Drawing [id:dp21753955847160644] 
\draw  [color={rgb, 255:red, 245; green, 166; blue, 35 }  ,draw opacity=1 ][line width=3] [line join = round][line cap = round] (184.26,189.14) .. controls (195.59,189.14) and (220.51,170.8) .. (230.53,165.35) .. controls (265.53,146.32) and (302.69,130.86) .. (339.6,114.92) .. controls (354.06,108.67) and (369.1,103.88) .. (382.56,95.89) .. controls (383.34,95.42) and (389.17,91.84) .. (389.17,91.13) ;
%Shape: Free Drawing [id:dp6240527636898042] 
\draw  [color={rgb, 255:red, 74; green, 144; blue, 226 }  ,draw opacity=1 ][line width=3] [line join = round][line cap = round] (174.34,96.84) .. controls (176.08,96.84) and (192.37,105.2) .. (196.37,106.35) .. controls (237.24,118.12) and (274.62,142.01) .. (313.16,162.5) .. controls (326.67,169.68) and (345.47,171.44) .. (359.43,178.68) .. controls (365.37,181.76) and (369.33,186.11) .. (374.85,189.14) .. controls (387.3,195.98) and (395.57,198.47) .. (405.7,207.23) ;
%Shape: Boxed Line [id:dp9017109815625127] 
\draw [color={rgb, 255:red, 208; green, 2; blue, 27 }  ,draw opacity=1 ]   (178.61,143) -- (316,141.56) ;
%Straight Lines [id:da5408328549080522] 
\draw    (165,114) -- (395.78,112.06) ;
%Straight Lines [id:da09197087660853087] 
\draw    (166,176) -- (398.03,174.8) ;
%Shape: Brace [id:dp7372954089935925] 
\draw   (415.16,175.98) .. controls (419.83,175.91) and (422.13,173.55) .. (422.06,168.88) -- (421.79,150.42) .. controls (421.69,143.75) and (423.97,140.39) .. (428.64,140.32) .. controls (423.97,140.39) and (421.59,137.09) .. (421.5,130.42)(421.54,133.42) -- (421.27,114.32) .. controls (421.2,109.65) and (418.83,107.35) .. (414.16,107.42) ;


% Text Node
\draw (433.47,130.38) node [anchor=north west][inner sep=0.75pt]   [align=left] {{\fontsize{0.67em}{0.8em}\selectfont \textbf{Spread \ BID-ASK}}};
% Text Node
\draw (122,104) node [anchor=north west][inner sep=0.75pt]   [align=left] {{\tiny Precio ASK}};
% Text Node
\draw (123,164) node [anchor=north west][inner sep=0.75pt]   [align=left] {{\tiny Precio BID}};
% Text Node
\draw (321,131) node [anchor=north west][inner sep=0.75pt]   [align=left] {{\tiny PRECIO EQUILIBRIO}};
% Text Node
\draw (398,77) node [anchor=north west][inner sep=0.75pt]   [align=left] {{\tiny OFERTA}};
% Text Node
\draw (407,202) node [anchor=north west][inner sep=0.75pt]   [align=left] {{\tiny DEMANDA}};


\end{tikzpicture}

\hypertarget{anuxe1lisis-de-la-contribuciuxf3n-del-trabajo-presentado}{%
\subsection{Análisis de la contribución del trabajo
presentado}\label{anuxe1lisis-de-la-contribuciuxf3n-del-trabajo-presentado}}

Los modelos econométricos presentados y su aplicabilidad civil y real
son en el general de la exposición la mayor contribución del trabajo
presentado, considerando que en todos los casos expuestos tienen como
información basal a variables reales e históricas, utilizadas para
predecir y en consecuencia a partir de estos datos históricos relizar el
modelamiento predictivo, se considera como una segunda contribución de
los trabajos, la componente social resultante derivada del modelamiento,
ya que a partir de sus resultados de los modelos econométricos, se puede
mejorar la calidad de vida, eficiencia del entorno donde se habita y las
condiciones de vida de la sociedad.

\hypertarget{comentario-adicional}{%
\subsection{Comentario adicional}\label{comentario-adicional}}

En general la aplición de esta rama de la economía condiciona el futuro
de las sociedades, estas herramientas son utilizadas en la toma de
deciciones transversales, estimando que estos modelos matemáticos son
utilizados como instrumentos de decisión, apoyando a mejorar en todo
ambito a las sociedades fundadas en las transaccines económicas. Se
considera importante que los modelos sean alimentados de información
histórica validada, correcta y actualizada, dada la dinámica no lineal
de la realidad.

\hypertarget{bibliografuxeda}{%
\subsection{Bibliografía}\label{bibliografuxeda}}

\printbibliography

\end{document}
