% Options for packages loaded elsewhere
\PassOptionsToPackage{unicode}{hyperref}
\PassOptionsToPackage{hyphens}{url}
%
\documentclass[
  american,
]{article}
\usepackage{lmodern}
\usepackage{amssymb,amsmath}
\usepackage{ifxetex,ifluatex}
\ifnum 0\ifxetex 1\fi\ifluatex 1\fi=0 % if pdftex
  \usepackage[T1]{fontenc}
  \usepackage[utf8]{inputenc}
  \usepackage{textcomp} % provide euro and other symbols
\else % if luatex or xetex
  \usepackage{unicode-math}
  \defaultfontfeatures{Scale=MatchLowercase}
  \defaultfontfeatures[\rmfamily]{Ligatures=TeX,Scale=1}
\fi
% Use upquote if available, for straight quotes in verbatim environments
\IfFileExists{upquote.sty}{\usepackage{upquote}}{}
\IfFileExists{microtype.sty}{% use microtype if available
  \usepackage[]{microtype}
  \UseMicrotypeSet[protrusion]{basicmath} % disable protrusion for tt fonts
}{}
\makeatletter
\@ifundefined{KOMAClassName}{% if non-KOMA class
  \IfFileExists{parskip.sty}{%
    \usepackage{parskip}
  }{% else
    \setlength{\parindent}{0pt}
    \setlength{\parskip}{6pt plus 2pt minus 1pt}}
}{% if KOMA class
  \KOMAoptions{parskip=half}}
\makeatother
\usepackage{xcolor}
\IfFileExists{xurl.sty}{\usepackage{xurl}}{} % add URL line breaks if available
\IfFileExists{bookmark.sty}{\usepackage{bookmark}}{\usepackage{hyperref}}
\hypersetup{
  pdftitle={refereeSEM\_DELAFUENTE},
  pdfauthor={Manuel Ayala},
  pdflang={en-US},
  hidelinks,
  pdfcreator={LaTeX via pandoc}}
\urlstyle{same} % disable monospaced font for URLs
\usepackage[margin=1in]{geometry}
\usepackage{graphicx}
\makeatletter
\def\maxwidth{\ifdim\Gin@nat@width>\linewidth\linewidth\else\Gin@nat@width\fi}
\def\maxheight{\ifdim\Gin@nat@height>\textheight\textheight\else\Gin@nat@height\fi}
\makeatother
% Scale images if necessary, so that they will not overflow the page
% margins by default, and it is still possible to overwrite the defaults
% using explicit options in \includegraphics[width, height, ...]{}
\setkeys{Gin}{width=\maxwidth,height=\maxheight,keepaspectratio}
% Set default figure placement to htbp
\makeatletter
\def\fps@figure{htbp}
\makeatother
\setlength{\emergencystretch}{3em} % prevent overfull lines
\providecommand{\tightlist}{%
  \setlength{\itemsep}{0pt}\setlength{\parskip}{0pt}}
\setcounter{secnumdepth}{-\maxdimen} % remove section numbering
\usepackage{pgf,tikz}
\ifxetex
  % Load polyglossia as late as possible: uses bidi with RTL langages (e.g. Hebrew, Arabic)
  \usepackage{polyglossia}
  \setmainlanguage[variant=american]{english}
\else
  \usepackage[shorthands=off,main=american]{babel}
\fi
\usepackage[]{biblatex}
\addbibresource{./references.bib}

\title{refereeSEM\_DELAFUENTE}
\usepackage{etoolbox}
\makeatletter
\providecommand{\subtitle}[1]{% add subtitle to \maketitle
  \apptocmd{\@title}{\par {\large #1 \par}}{}{}
}
\makeatother
\subtitle{\textbf{Network Design Problems with traffic Capture}}
\author{Manuel Ayala}
\date{14 julio 2020}

\begin{document}
\maketitle

\hypertarget{descripciuxf3n-del-trabajo-presentado}{%
\subsection{Descripción del trabajo
presentado}\label{descripciuxf3n-del-trabajo-presentado}}

El seminario realizado por el Dr.~De La Fuente presentó modelos
econométricos con carácter social. El desarrollo de la exposición
consideró tres casos de estudios definidos como:

\begin{itemize}
\tightlist
\item
  Caso 1 Analysis of the factor of Chilean city hall using econometric
  modeling and stochastic frontier.
\end{itemize}

Considerando el índice de calidad de vida \((QoL)\) sectorizado por
comunas de Chile, publicado por la PUC en el año 2018, el cual
consideraba variables cualitativas y cuantitativas de la calidad como
conectividad, habitabilidad, salud, condiciones sociocultirales y otras,
se desarrollo estudio que buscaba determinar el grado de influencia real
de la calidad de vida por comuna y estimar la eficiencia de las
municipalidades a partir de los indicadores de calidad de vida como una
calificación de vida urbana. Esto se relizó basado en estadísticas
descriptivas y el modelo de regresión \emph{stepwise}, análisis de
cluster y stochastic frontier model

Cruzando información de bases de datos existentes se determinó por la
utilización de bases de información cuantitativa y también cualitativa,
que buscaba al encasillar estos modelos en términos sociales los
aterriza en la civilidad y en la

\hypertarget{anuxe1lisis-de-la-literatura-relacionada}{%
\subsection{Análisis de la literatura
relacionada}\label{anuxe1lisis-de-la-literatura-relacionada}}

\includegraphics{Figs/unnamed-chunk-2-1.pdf}
\includegraphics{Figs/unnamed-chunk-2-2.pdf}
\includegraphics{Figs/unnamed-chunk-2-3.pdf}
\includegraphics{Figs/unnamed-chunk-2-4.pdf}
\includegraphics{Figs/unnamed-chunk-2-5.pdf}

\includegraphics{Figs/unnamed-chunk-3-1.pdf}

\includegraphics{Figs/unnamed-chunk-4-1.pdf}

\hypertarget{descripciuxf3n-del-marco-conceptual}{%
\subsection{Descripción del marco
conceptual}\label{descripciuxf3n-del-marco-conceptual}}

\hypertarget{diagrams}{%
\subsection{Diagrams}\label{diagrams}}

\tikzset{every picture/.style={line width=0.75pt}}

\%set default line width to 0.75pt

\begin{tikzpicture}[x=0.75pt,y=0.75pt,yscale=-1,xscale=1]
%uncomment if require: \path (0,300); %set diagram left start at 0, and has height of 300

%Shape: Axis 2D [id:dp19527234072026634] 
\draw  (151.2,238.94) -- (430.21,238.65)(178.8,44.76) -- (179.14,260.49) (423.2,233.66) -- (430.21,238.65) -- (423.22,243.66) (173.81,51.77) -- (178.8,44.76) -- (183.81,51.76)  ;
%Shape: Free Drawing [id:dp21753955847160644] 
\draw  [color={rgb, 255:red, 245; green, 166; blue, 35 }  ,draw opacity=1 ][line width=3] [line join = round][line cap = round] (184.26,189.14) .. controls (195.59,189.14) and (220.51,170.8) .. (230.53,165.35) .. controls (265.53,146.32) and (302.69,130.86) .. (339.6,114.92) .. controls (354.06,108.67) and (369.1,103.88) .. (382.56,95.89) .. controls (383.34,95.42) and (389.17,91.84) .. (389.17,91.13) ;
%Shape: Free Drawing [id:dp6240527636898042] 
\draw  [color={rgb, 255:red, 74; green, 144; blue, 226 }  ,draw opacity=1 ][line width=3] [line join = round][line cap = round] (174.34,96.84) .. controls (176.08,96.84) and (192.37,105.2) .. (196.37,106.35) .. controls (237.24,118.12) and (274.62,142.01) .. (313.16,162.5) .. controls (326.67,169.68) and (345.47,171.44) .. (359.43,178.68) .. controls (365.37,181.76) and (369.33,186.11) .. (374.85,189.14) .. controls (387.3,195.98) and (395.57,198.47) .. (405.7,207.23) ;
%Shape: Boxed Line [id:dp9017109815625127] 
\draw [color={rgb, 255:red, 208; green, 2; blue, 27 }  ,draw opacity=1 ]   (178.61,143) -- (316,141.56) ;
%Straight Lines [id:da5408328549080522] 
\draw    (165,114) -- (395.78,112.06) ;
%Straight Lines [id:da09197087660853087] 
\draw    (166,176) -- (398.03,174.8) ;
%Shape: Brace [id:dp7372954089935925] 
\draw   (415.16,175.98) .. controls (419.83,175.91) and (422.13,173.55) .. (422.06,168.88) -- (421.79,150.42) .. controls (421.69,143.75) and (423.97,140.39) .. (428.64,140.32) .. controls (423.97,140.39) and (421.59,137.09) .. (421.5,130.42)(421.54,133.42) -- (421.27,114.32) .. controls (421.2,109.65) and (418.83,107.35) .. (414.16,107.42) ;


% Text Node
\draw (433.47,130.38) node [anchor=north west][inner sep=0.75pt]   [align=left] {{\fontsize{0.67em}{0.8em}\selectfont \textbf{Spread \ BID-ASK}}};
% Text Node
\draw (122,104) node [anchor=north west][inner sep=0.75pt]   [align=left] {{\tiny Precio ASK}};
% Text Node
\draw (123,164) node [anchor=north west][inner sep=0.75pt]   [align=left] {{\tiny Precio BID}};
% Text Node
\draw (321,131) node [anchor=north west][inner sep=0.75pt]   [align=left] {{\tiny PRECIO EQUILIBRIO}};
% Text Node
\draw (398,77) node [anchor=north west][inner sep=0.75pt]   [align=left] {{\tiny OFERTA}};
% Text Node
\draw (407,202) node [anchor=north west][inner sep=0.75pt]   [align=left] {{\tiny DEMANDA}};


\end{tikzpicture}

\hypertarget{anuxe1lisis-de-la-contribuciuxf3n-del-trabajo-presentado}{%
\subsection{Análisis de la contribución del trabajo
presentado}\label{anuxe1lisis-de-la-contribuciuxf3n-del-trabajo-presentado}}

\hypertarget{comentario-adicional}{%
\subsection{Comentario adicional}\label{comentario-adicional}}

\hypertarget{bibliografuxeda}{%
\subsection{Bibliografía}\label{bibliografuxeda}}

\printbibliography

\end{document}
